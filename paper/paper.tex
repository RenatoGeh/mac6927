\documentclass{amsart}

% biber

\usepackage[english]{babel}
\usepackage[utf8]{inputenc}
\usepackage{graphicx}
\usepackage{mathtools}
\usepackage{amsthm}
\usepackage{thmtools,thm-restate}
\usepackage{amsfonts}
\usepackage{hyperref}
\usepackage[backend=biber,url=true,doi=true,eprint=false,style=alphabetic]{biblatex}
\usepackage{enumitem}
\usepackage[justification=centering,singlelinecheck=false]{caption}
\usepackage{indentfirst}
\usepackage{algorithm}
\usepackage{algpseudocode}
\usepackage{listings}
\usepackage[x11names, rgb]{xcolor}
\usepackage{tikz}
\usepackage{hyperref}
\usepackage{subcaption}
\usepackage{booktabs}
\usepackage{linegoal}
\usepackage{csquotes}
\usetikzlibrary{snakes,arrows,shapes}

\addbibresource{references.bib}

\makeatletter
\def\subsection{\@startsection{subsection}{3}%
  \z@{.5\linespacing\@plus.7\linespacing}{.1\linespacing}%
  {\normalfont}}
\makeatother

\makeatletter
\patchcmd{\@setauthors}{\MakeUppercase}{}{}{}
\makeatother

\DeclareMathOperator*{\argmin}{arg\,min}
\DeclareMathOperator*{\argmax}{arg\,max}
\DeclareMathOperator*{\Val}{\text{Val}}
\DeclareMathOperator*{\Ch}{\text{Ch}}
\DeclareMathOperator*{\Pa}{\text{Pa}}
\DeclareMathOperator*{\Sc}{\text{Sc}}
\newcommand{\ov}{\overline}
\newcommand{\region}{\mathcal}

\newcommand\defeq{\mathrel{\overset{\makebox[0pt]{\mbox{\normalfont\tiny\sffamily def}}}{=}}}

\newcommand{\algorithmautorefname}{Algorithm}
\algrenewcommand\algorithmicrequire{\textbf{Input}}
\algrenewcommand\algorithmicensure{\textbf{Output}}
\algrenewcomment[1]{\hspace{0.25cm}\(\triangleright\) #1}
\algnewcommand{\LineComment}[1]{\State\,\(\triangleright\) #1}

\captionsetup[table]{labelsep=space}

\setlistdepth{9}
\newlist{deepitemize}{itemize}{9}
\setlist[deepitemize,1]{label=$\bullet$}
\setlist[deepitemize,2]{label=$\bullet$}
\setlist[deepitemize,3]{label=$\bullet$}
\setlist[deepitemize,4]{label=$\bullet$}
\setlist[deepitemize,5]{label=$\bullet$}
\setlist[deepitemize,6]{label=$\bullet$}
\setlist[deepitemize,7]{label=$\bullet$}
\setlist[deepitemize,8]{label=$\bullet$}
\setlist[deepitemize,9]{label=$\bullet$}

\theoremstyle{plain}

\newcounter{dummy-def}\numberwithin{dummy-def}{section}
\newtheorem{definition}[dummy-def]{Definition}
\newcounter{dummy-thm}\numberwithin{dummy-thm}{section}
\newtheorem{theorem}[dummy-thm]{Theorem}
\newcounter{dummy-prop}\numberwithin{dummy-prop}{section}
\newtheorem{proposition}[dummy-prop]{Proposition}
\newcounter{dummy-corollary}\numberwithin{dummy-corollary}{section}
\newtheorem{corollary}[dummy-corollary]{Corollary}
\newcounter{dummy-lemma}\numberwithin{dummy-lemma}{section}
\newtheorem{lemma}[dummy-lemma]{Lemma}
\newcounter{dummy-ex}\numberwithin{dummy-ex}{section}
\newtheorem{exercise}[dummy-ex]{Exercise}
\newcounter{dummy-eg}\numberwithin{dummy-eg}{section}
\newtheorem{example}[dummy-eg]{Example}

\numberwithin{equation}{section}

\newcommand{\set}[1]{\mathbf{#1}}
\newcommand{\pr}{\mathbb{P}}
\newcommand{\eps}{\varepsilon}
\renewcommand{\implies}{\Rightarrow}

\newcommand{\bigo}{\mathcal{O}}

\setlength{\parskip}{1em}

\lstset{frameround=fttt,
	numbers=left,
	breaklines=true,
	keywordstyle=\bfseries,
	basicstyle=\ttfamily,
}

\newcommand{\code}[1]{\lstinline[mathescape=true]{#1}}
\newcommand{\mcode}[1]{\lstinline[mathescape]!#1!}
\newcommand{\dset}[1]{\mathcal{#1}}
\newcommand{\ddspn}[2]{\frac{\partial#1}{\partial#2}}
\newcommand{\iddspn}[2]{\partial#1/\partial#2}

\title{%
  \noindent\rule{13cm}{1.0pt}\\
  \vspace{0.2cm}
  A Polynomial-time Reduction of the Quadratic Congruence Problem to the 3-SAT and other Related Problems
  \noindent\rule{13cm}{0.8pt}
}
\xdef\shorttitle{Quadratic congruence polynomial-time reduction}
\author[]{\normalsize Renato Lui Geh\\\small NUSP\@: 8536030\\\\Computational Number Theory
(MAC6927)\\Prof\@. Sinai Robins\\University of São Paulo\\}

\begin{document}

\begin{abstract}
  In this term paper for MAC6927 --- Computational Number Theory, we explore the history behind the
  quadratic congruence problem (QCP) and other related number theoric problems; show a
  polynomial-time reduction from the QCP to the 3-SAT quoting Adleman and Mander's 1978
  theorem~\cite{qcp2}, implying that quadratic congruence is NP-complete; and show some solved and
  unsolved problems in Number Theory that are directly (or indirectly) related to the QCP problem
  and its membership in NP\@.
  \vspace*{-3.5em}
\end{abstract}

\maketitle

\section{History}

German mathematician David Hilbert published~\cite{hilbert} in 1902 a set of 23 unsolved problems
in mathematics he deemed to be most important unanswered mathematical problems of the 20th century.
Since then 9 of them have been solved (at the time the author is writing this line and as far as
the author is aware), 9 are considered partially solved, three of them are unsolved and two of them
are considered too vague. Unsolved problems include the infamous Riemann Hypothesis and an
extension to the Kronecker-Weber Theorem. Amongst solved problems is the 10th Hilbert problem.

\textbf{10th Hilbert Problem:} Given a diophantine equation with any number of unknown quantities
and with rational integral numerical coefficients: \textit{to devise a process according to which
it can be determined by a finite number of operations whether the equation is solvable in rational
integers}.

It was answered in 1970 by Matiyasevich~\cite{diophantine} to be impossible. The question now
becomes, in which cases is there an algorithm for solvability and what is the complexity of such
algorithms? In 1976, Adleman and Manders~\cite{qcp1} partially answered these questions by proving
that, for the quadratic cases, there exists an algorithm and it is NP-complete. In their proof,
they also found that, through a slight modification in the final step of their proof, it was
possible to answer the quadratic congruence problem. A cleaner version of this proof was published
in 1978 by Adleman and Manders~\cite{qcp2}, proof we try to explain in this paper.

In this paper we focus on the second result of Adleman and Mander's 1978 article, but also show the
main result, namely that the set of quadratic diophantine equations with natural numbers solutions
is NP-complete. The proof is done through a polynomial-time reduction to the 3-SAT problem. This
reduction implies that both problems covered in Adleman and Mander's article are in NP-complete.

In 1971, American mathematician Stephen Cook published ``The Complexity of Theorem-proving
Procedures''~\cite{cook}, and in the next year, his fellow countryman Richard Karp published
``Reducibility Among Combinatorial Problems''~\cite{karp}. The two articles introduced the concepts
of P and NP classes, yielding the duo a Turing Award. Interestingly in 1973, on the other side of
the Iron Curtain, Ukrainian Leonard Levin published~\cite{levin} in the USSR equivalent results to
Cook's and Karp's, but considering search problems instead of decision problems (an interesting
remark is that Levin did not receive a Turing Award for his work, despite having achieved
equivalent results). Both works resulted in the following statement: that any problem in NP can be
reduced in polynomial time by a deterministic Turing machine to the problem of satisfiability of a
Boolean formula, i.e.\ the SAT problem.  Additionally, if there exists a deterministic polynomial
time algorithm for solving SAT, then every NP problem can be solved by a deterministic polynomial
time algorithm.

This independent, parallel work from opposite parts of the world, ideologically and geographically,
gave rise to what is considered one of the most important open questions in theoretical computer
science, the P vs NP problem.

\begin{deepitemize}
  \item Satisfiability (SAT)
    \begin{deepitemize}
      \item 0--1 integer programming
      \item Clique
        \begin{deepitemize}
          \item Set packing
          \item Vertex cover
            \begin{deepitemize}
              \item Set covering
              \item Feedback node set
              \item Feedback arc set
              \item Directed Hamiltonian cycle
                \begin{deepitemize}
                  \item Undirected Hamiltonian cycle
                \end{deepitemize}
            \end{deepitemize}
        \end{deepitemize}
      \item Satisfiability with at most 3 literals per clause (3-SAT)
        \begin{deepitemize}
          \item Chromatic number
            \begin{deepitemize}
              \item Clique cover
              \item Exact cover
                \begin{deepitemize}
                  \item Hitting set
                  \item Steiner set
                  \item 3-dimensional matching
                  \item Knapsack
                    \begin{deepitemize}
                      \item Job sequencing
                      \item Partition
                        \begin{deepitemize}
                          \item Max cut
                        \end{deepitemize}
                    \end{deepitemize}
                \end{deepitemize}
            \end{deepitemize}
        \end{deepitemize}
    \end{deepitemize}
\end{deepitemize}

In his 1972 paper, Karp also published a list of 21 NP-complete problems in which he showed
reductions implying its membership in the NP-complete class. The list above is Karp's 21
NP-complete problems, where the nesting indicates the direction of the reduction. For instance, the
knapsack problem was reduced to the exact cover problem, which was reduced to chromatic number,
which was reduced to the 3-SAT, and the 3-SAT was reduced to the SAT problem.

%--------------------------------------------------------------------------------------------------

\printbibliography[]

\end{document}
